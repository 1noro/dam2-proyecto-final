\documentclass[12pt,a4paper,titlepage]{article}
\usepackage[utf8]{inputenc}
\usepackage[spanish]{babel}

% para que el índice y las referencias tengan enlaces
\usepackage[hidelinks]{hyperref}

% mostrar colores
\usepackage{xcolor}
% mostrar código
\usepackage{listings}
\renewcommand{\lstlistingname}{Ejemplo}
\renewcommand{\lstlistlistingname}{Ejemplos de código}
% estilo de código para JavaScript
% (https://www.youtube.com/watch?v=qxkQgG1Y0bY)
\lstdefinestyle{javastyle}{
    basicstyle=\ttfamily\small,
    breaklines=true,
    commentstyle=\color[HTML]{2b922c},
    keywordstyle=\color[HTML]{6b2eff},
    stringstyle=\color[HTML]{ac2f38},
    showstringspaces=false,
    numbers=left,
    numberstyle=\ttfamily\footnotesize,
    stepnumber=1
}

\title{PROYECTO FINAL D.A.M.\\\emph{\small{(nombre provisional)}}}
\author{Alfredo Rodríguez Gracía}

\begin{document}
    \maketitle
    \tableofcontents
    %\hrule
    \lstlistoflistings
    \newpage

    \section{Descripción del proyecto}

    Breve descripción del proyecto indicando qué es lo que hace, para qué sirve y si tiene futuros usos con pequeñas modificaciones.

    Como vemos en el ejemplo \ref{example1}, las \emph{arrow functions} son preciosas.

    \section{Ámbito de implantación}

    Deberá describirse el lugar (empresa, organización, sector...) en el que se implantará el proyecto y con qué objetivo, además de indicar a quién va dirigida la aplicación, es decir, identificar quién o quiénes serán los principales usuarios de la misma.

    \section{Recursos de hardware y software}

    Se describirán los requisitos mínimos y los requisitos recomendados de hardware, tanto para el desarrollo de la aplicación, como para su instalación y ejecución.

    Se describirán las necesidades de software requeridas para el desarrollo de la aplicación.

    \section{Temporalización del desarrollo}

    Deben describirse las distintas actividades necesarias para desarrollar el proyecto, asignarles un tiempo a cada una de ellas y construir los dos diagramas completos.

    \subsection{Diagrama de Gantt}

    \dots

    \subsection{Diagrama PERT}

    \dots

    \section{Descripción de los datos base y resultados}

    Se describirán el tipo de campo (en caso de java serían: String, char, int, double, long\dots), que se utilizará para recoger los diferentes datos.

    Posibles restricciones y/o estructuras utilizadas (clases). Lo mismo para los datos resultantes de los procesos.

    \section{Relación entre dispositivos y programa o rutinas}

    Se identificarán los componentes que comunican el paquete o aplicación software desarrollado con el resto de actores relevantes fuera de la máquina. Es decir, interfaces persona-máquina para entrada y/o salida de datos, interfaces de red u otros medios para comunicación con máquinas remotas, periféricos específicos o componentes concretos de plataformas móviles, etc.

    Se identificarán los componentes software (clases, procedimientos) representativos y se vincularán con los anteriormente mencionados a través de texto y/o diagrama(s) que ayuden a comprender el funcionamiento general de la aplicación.

    \newpage

    \lstinputlisting[language=Java, style=javastyle, caption=Las funciones de ejemplo, label=example1]{code/example1.js}

\end{document}
